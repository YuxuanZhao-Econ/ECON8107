\documentclass[11pt]{article}
\usepackage[margin=1in]{geometry}
\usepackage{amsmath,amsthm,amssymb,mathtools}
\usepackage[margin=1in]{geometry}  % already defined in main document
\usepackage{amsmath,amsthm,amssymb,mathtools}  % already defined






\title{Econ8107 Assignment 3}
\author{Yuxuan Zhao}
\date{}
\begin{document}
\maketitle

\section*{Question 1}

\subsection*{Part (a)}
% ----------------------------
% Pset 3, Q1(a): Lowest kink in Huggett asset supply
% ----------------------------

\paragraph{Household problem (Huggett).}
Given the gross interest rate $R$ and an idiosyncratic income state $s\in S$
following a Markov chain with transition $\pi(s'|s)$, the Bellman equation is
\begin{align*}
V(a,s)&=\max_{a'\ge -\phi}\left\{
u\!\big(c\big)+\beta\sum_{s'}\pi(s'|s)V(a',s')
\right\}, \quad \text{s.t.} \\
c &= y(s)+Ra-a' \\
a' &\ge -\phi
\end{align*}
Let $\mu\ge 0$ be the multiplier on the borrowing constraint $a'\ge -\phi$, we have FOC:
\begin{align*}
u'(c)&=\beta R \sum_{s'}\pi(s'|s)\,V_a(a',s')+\mu, \\
\mu&\ge 0,\quad a'+\phi\ge 0,\quad \mu(a'+\phi)=0.
\end{align*}
Using the envelope condition $V_a(a',s')=u'(c(s'))$, we obtain the Euler inequality
\begin{align*}
u'(c(s)) \ \ge\ \beta R \sum_{s'}\pi(s'|s)\,u'(c(s')),\qquad \forall s\in S,
\end{align*}
and the borrowing constraint binds ($a'=-\phi$) if and only if
\begin{align*}
&u'(c(s)) \ >\ \beta R \sum_{s'}\pi(s'|s)\,u'(c(s')) \\
\Longleftrightarrow &R<\frac{1}{\beta}\frac{u'(c(s))}{\sum_{s'}\pi(s'|s)\,u'(c(s'))}.
\end{align*}

If $a'=-\phi$, then
\[
c(s)=y(s)+R(-\phi)-(-\phi)=y(s)-(R-1)\phi.
\]

For aggregate saving to equal $-\phi$, all households must choose $a'=-\phi$ in every state:
\[
u'\!\big(y(s)-(R-1)\phi\big)\ \ge\ \beta R\sum_{s'}\pi(s'|s)\,
u'\!\big(y(s')-(R-1)\phi\big),\qquad \forall s\in S.
\]

Equivalently, $\bar R(\phi)$ is characterized (implicitly) by the tightest state:
\[
\boxed{
\bar R(\phi)=\frac{1}{\beta}\min_{s\in S}
\frac{u'\!\big(y(s)-(\bar R(\phi)-1)\phi\big)}
{\sum_{s'}\pi(s'|s)\,u'\!\big(y(s')-(\bar R(\phi)-1)\phi\big)}
}
\]
The interest rate $\bar R(\phi)$ depends on the borrowing limit $\phi$.

For $R=\bar R(\phi)$, there exists a household $s^*$ which is indifferent between borrowing $-\phi$ and saving more, and all other households strictly prefer to borrow $-\phi$. 



\subsection*{Part (c)}

We consider the two-state case $s\in\{s_1,s_2\}$ and write $y_i:=y(s_i)$ and $\pi_i:=\pi(s_i)$ for $i=1,2$.

If the household chooses $a'=-\phi$, then:
\[
c_i(R)=y_i-(R-1)\phi=y_i+(1-R)\phi,\qquad i=1,2.
\]

\paragraph{Euler inequality under log utility and i.i.d.\ income.}
Since $u'(c)=1/c$, the Euler condition with a binding borrowing constraint is
\[
\frac{1}{c_i(R)} \ \ge\ \beta R\sum_{j=1}^2 \pi_j \frac{1}{c_j(R)},\qquad i=1,2,
\]

Plug in $c_i(R)=y_i+(1-R)\phi$, we have
\[
\frac{1}{y_i+(1-R)\phi} \ \ge\ \beta R\sum_{j=1}^2 \pi_j \frac{1}{y_j+(1-R)\phi},\qquad i=1,2.
\]

The right-hand side does not depend on the current state, we only need borrowing constraint to bind for the high-income household (so they don't want to save more)

Suppose $y_1<y_2$, then:
\[
\frac{1}{y(s_1)-(\bar R-1)\phi} >\frac{1}{y(s_2)-(\bar R-1)\phi},
\]

Then we know that $R$ need to satisfy the Euler inequality for the high-income household:
\[
\boxed{\frac{1}{y_2+(1-R)\phi} \ \ge\ \beta R\sum_{j=1}^2 \pi_j \frac{1}{y_j+(1-R)\phi}}.
\]
We can rewrite the above inequality as:
\begin{align*}
\beta R \leq \frac{\frac{1}{y_2+(1-R)\phi}}{\sum_{j=1}^2 \pi_j \frac{1}{y_j+(1-R)\phi}}.
\end{align*}

Define
\[
H(R)\;:=\;\frac{\frac{1}{y_2+(1-R)\phi}}{\sum_{j=1}^2 \pi_j \frac{1}{y_j+(1-R)\phi}}.
\]

We know that there exist a finite $\bar R$ satisfying $H(\bar R)=\beta \bar R$.
\footnote{
We have $\bar R$ satisfies:
\begin{align*}
\bar R = \frac{1}{\beta}\min_{s\in S}
\frac{\frac{1}{y(s)-(\bar R-1)\phi}}
{\pi(s_1) \frac{1}{y(s_1)-(\bar R-1)\phi} + \pi(s_2) \frac{1}{y(s_2)-(\bar R-1)\phi}}
\end{align*}
Suppose $y_1<y_2$, then:
\[
\frac{1}{y(s_1)-(\bar R-1)\phi} >\frac{1}{y(s_2)-(\bar R-1)\phi},
\]
we only need borrowing constraint to bind for the high-income household (so they don't want to save more)

Then $\bar R$ satisfies:
\[
\bar R = \frac{1}{\beta}\frac{\frac{1}{y(s_2)-(\bar R-1)\phi}}
{\pi(s_1) \frac{1}{y(s_1)-(\bar R-1)\phi} + \pi(s_2) \frac{1}{y(s_2)-(\bar R-1)\phi}}.
\]
We can obtain a finite solution $\bar R$ to the above equation.}


We want to show that for any $R\in[0,\bar R]$, we have $H(R)\ge \beta R$.

Let
\[
a_j(R):=\frac{1}{y_j+(1-R)\phi},\qquad S(R):=\sum_{j=1}^2\pi_j a_j(R),
\]
so that $H(R)=\dfrac{a_2(R)}{S(R)}$. Note that
\[
a_j'(R)=\frac{d}{dR}\Big(\frac{1}{y_j+(1-R)\phi}\Big)
=\frac{\phi}{\big(y_j+(1-R)\phi\big)^2}=\phi\,a_j(R)^2>0,
\]
and hence
\[
S'(R)=\sum_{j=1}^2 \pi_j a_j'(R)=\phi\sum_{j=1}^2 \pi_j a_j(R)^2.
\]
By the quotient rule,
\begin{align*}
H'(R)
&=\frac{a_2'(R)S(R)-a_2(R)S'(R)}{S(R)^2}\\
&=\frac{\phi a_2(R)^2 S(R)-\phi a_2(R)\sum_{j=1}^2\pi_j a_j(R)^2}{S(R)^2}\\
&=\frac{\phi a_2(R)}{S(R)^2}\left(a_2(R)S(R)-\sum_{j=1}^2\pi_j a_j(R)^2\right)\\
&=\frac{\phi a_2(R)}{S(R)^2}\sum_{j=1}^2\pi_j a_j(R)\big(a_2(R)-a_j(R)\big).
\end{align*}
Since $y_2>y_1$ implies $y_2+(1-R)\phi>y_1+(1-R)\phi$ and thus $a_2(R)<a_1(R)$, we have
\[
\sum_{j=1}^2\pi_j a_j(R)\big(a_2(R)-a_j(R)\big)
=\pi_1 a_1(R)\big(a_2(R)-a_1(R)\big)+\pi_2 a_2(R)\big(a_2(R)-a_2(R)\big)<0,
\]
so $H'(R)<0$, i.e.\ $H(R)$ is strictly decreasing in $R$.

Since $\beta R$ is strictly increasing in $R$ and $H(\bar R)=\beta \bar R$, it follows that for any
$R\in[0,\bar R]$,
\[
H(R)\ge H(\bar R)=\beta \bar R \ge \beta R.
\]
Therefore, for any $R\in[0,\bar R]$ the tight-state Euler inequality holds, hence it holds for both states,
and the borrowing constraint binds for everyone: $a'(s_i)=-\phi$ for $i=1,2$. Consequently,
\[
A(R)=\sum_{i=1}^2\pi_i a'(s_i)=\sum_{i=1}^2\pi_i(-\phi)=-\phi,\qquad \forall\,R\in[0,\bar R].
\]


\subsection*{Part (d)}

Suppose income is i.i.d., i.e.\ $\pi(s'|s)=\pi(s')$, with a finite support $S=\{s_1,\dots,s_N\}$ and
associated endowments $y_1<y_2<\cdots<y_N$, where $y_k:=y(s_k)$ and $\pi_k:=\pi(s_k)$.
Assume CRRA utility $u(c)=\frac{c^{1-\gamma}}{1-\gamma}$ with $\gamma>0$ (so $u'(c)=c^{-\gamma}$), and
$\phi<\min_k y_k$.

\paragraph{Candidate borrowing-limit allocation.}
If the household chooses $a'=-\phi$, then in stationarity $a=-\phi$ and consumption in state $s_k$ is
\[
c_k(R)=y_k+R(-\phi)-(-\phi)=y_k-(R-1)\phi=y_k+(1-R)\phi,\qquad k=1,\dots,N,
\]
which is strictly positive on the relevant range.

\paragraph{Euler inequality and the tight state.}
With CRRA, $u'(c)=c^{-\gamma}$. Under i.i.d.\ income, the Euler condition at the borrowing limit is
\[
c_k(R)^{-\gamma}\ \ge\ \beta R\sum_{m=1}^N \pi_m\,c_m(R)^{-\gamma},\qquad k=1,\dots,N.
\]
The right-hand side does not depend on the current state $k$. Since $y_N$ is the highest endowment,
we have $c_N(R)>c_k(R)$ for all $k<N$, hence $c_N(R)^{-\gamma}<c_k(R)^{-\gamma}$. Therefore the
tightest inequality is for the highest-income state $s_N$.

Thus it suffices to require
\[
\boxed{
c_N(R)^{-\gamma}\ \ge\ \beta R\sum_{m=1}^N \pi_m\,c_m(R)^{-\gamma}.
}
\]
Equivalently,
\[
\beta R \ \le\ \frac{c_N(R)^{-\gamma}}{\sum_{m=1}^N \pi_m\,c_m(R)^{-\gamma}},
\qquad
H(R):=\frac{\big(y_N+(1-R)\phi\big)^{-\gamma}}{\sum_{m=1}^N \pi_m\big(y_m+(1-R)\phi\big)^{-\gamma}}.
\]

\paragraph{Monotonicity of $H(R)$.}
Let $b_m(R):=c_m(R)^{-\gamma}=\big(y_m+(1-R)\phi\big)^{-\gamma}$ and $B(R):=\sum_{m=1}^N \pi_m b_m(R)$,
so $H(R)=b_N(R)/B(R)$. Since $c_m'(R)=-\phi$, we have
\[
b_m'(R)=\frac{d}{dR}\big(c_m(R)^{-\gamma}\big)
=-\gamma c_m(R)^{-\gamma-1}\,c_m'(R)
=\gamma\phi\,c_m(R)^{-\gamma-1}>0,
\]
and hence $B'(R)=\sum_{m=1}^N \pi_m b_m'(R)>0$. By the quotient rule,
\[
H'(R)=\frac{b_N'(R)B(R)-b_N(R)B'(R)}{B(R)^2}
=\frac{1}{B(R)^2}\sum_{m=1}^N \pi_m\Big(b_N'(R)b_m(R)-b_N(R)b_m'(R)\Big).
\]
Now note that
\begin{align*}
b_N'(R)b_m(R)-b_N(R)b_m'(R)
&=\gamma\phi\,c_N(R)^{-\gamma-1}c_m(R)^{-\gamma}
-\gamma\phi\,c_N(R)^{-\gamma}c_m(R)^{-\gamma-1} \\
&=\gamma\phi\,c_N(R)^{-\gamma-1}c_m(R)^{-\gamma-1}\big(c_m(R)-c_N(R)\big).
\end{align*}
For $m<N$, we have $c_m(R)<c_N(R)$, so each term is strictly negative; for $m=N$ it is zero. Therefore
$H'(R)<0$, i.e.\ $H(R)$ is strictly decreasing in $R$.

\paragraph{Conclusion: a lowest kink on $[0,\bar R]$.}
Since $\beta R$ is strictly increasing in $R$ and $H(R)$ is strictly decreasing, there exists at most one
$\bar R$ such that $H(\bar R)=\beta\bar R$; define $\bar R$ as that intersection (when it exists).
Then for any $R\in[0,\bar R]$ we have $H(R)\ge \beta R$, so the tight-state Euler inequality holds, hence
it holds in all states and the borrowing constraint binds for everyone: $a'(s_k)=-\phi$ for all $k$.
Consequently, aggregate saving is constant on this interval:
\[
A(R)=\sum_{k=1}^N \pi_k\,a'(s_k)=\sum_{k=1}^N \pi_k(-\phi)=-\phi,
\qquad \forall\,R\in[0,\bar R].
\]




\end{document}

